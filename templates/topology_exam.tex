% Topology Exam
\documentclass[12pt]{article}
\usepackage{graphicx}
\usepackage[margin=1in]{geometry}
\usepackage{enumitem}
\usepackage{amsmath, amssymb, amsthm}

%-----------------------------------------------------------------
% Header with two logos side‑by‑side
%-----------------------------------------------------------------
\newcommand{\examheader}{
  \noindent
  \begin{minipage}[c]{0.25\textwidth}
    \centering
    % Replace with actual logo path
    \includegraphics[width=0.9\linewidth]{/home/mathxro/.gemini/antigravity/brain/c9d19247-36aa-4bd4-8a32-512063437249/logo1_placeholder_1763810843086.png}
  \end{minipage}%
  \hfill
  \begin{minipage}[c]{0.45\textwidth}
    \centering
    {\Large \textbf{Topology Exam}}\\[0.2cm]
    {\large Fall 2025}\\[0.2cm]
    {\large Date: \today}
  \end{minipage}%
  \hfill
  \begin{minipage}[c]{0.25\textwidth}
    \centering
    % Replace with actual logo path
    \includegraphics[width=0.9\linewidth]{/home/mathxro/.gemini/antigravity/brain/c9d19247-36aa-4bd4-8a32-512063437249/logo2_placeholder_1763810894171.png}
  \end{minipage}
  \vspace{0.5cm}
}

%-----------------------------------------------------------------
% Document start
%-----------------------------------------------------------------
\begin{document}
\examheader

\vspace{0.5cm}

%-----------------------------------------------------------------
% Exercise 1: Topological Spaces
%-----------------------------------------------------------------
\section*{Exercise 1 (Topological Spaces)}
Let $X = \{a, b, c\}$.
\begin{enumerate}[label=\alph*)]
  \item List all possible topologies on $X$ that contain the singleton set $\{a\}$.
  \item For one of the topologies found in part (a) that is not the discrete topology, determine the closure of the set $\{b\}$.
\end{enumerate}

%-----------------------------------------------------------------
% Exercise 2: Basis
%-----------------------------------------------------------------
\section*{Exercise 2 (Basis)}
Consider the real line $\mathbb{R}$ with the standard topology.
\begin{enumerate}[label=\alph*)]
  \item Define what it means for a collection $\mathcal{B}$ of subsets of $X$ to be a basis for a topology on $X$.
  \item Show that the collection of open intervals $(q_1, q_2)$ where $q_1, q_2 \in \mathbb{Q}$ is a basis for the standard topology on $\mathbb{R}$.
\end{enumerate}

%-----------------------------------------------------------------
% Exercise 3: Continuity
%-----------------------------------------------------------------
\section*{Exercise 3 (Continuity)}
Let $X$ and $Y$ be topological spaces and $f: X \to Y$ be a function.
\begin{enumerate}[label=\alph*)]
  \item State the definition of continuity in terms of open sets.
  \item Prove that $f$ is continuous if and only if for every closed set $C \subseteq Y$, the preimage $f^{-1}(C)$ is closed in $X$.
\end{enumerate}

%-----------------------------------------------------------------
% Exercise 4: Compactness
%-----------------------------------------------------------------
\section*{Exercise 4 (Compactness)}
\begin{enumerate}[label=\alph*)]
  \item Define what it means for a topological space $X$ to be compact.
  \item Prove that a closed subset of a compact space is compact.
\end{enumerate}

%-----------------------------------------------------------------
% Exercise 5: Connectedness
%-----------------------------------------------------------------
\section*{Exercise 5 (Connectedness)}
Let $X$ be a topological space.
\begin{enumerate}[label=\alph*)]
  \item Define what it means for $X$ to be connected.
  \item Let $A$ be a connected subset of $X$. Prove that if $A \subseteq B \subseteq \overline{A}$, then $B$ is connected.
\end{enumerate}

\end{document}
